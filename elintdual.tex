\documentclass{article}
\title{ALO for Elastic Net with Intercept through Generalized LASSO}
\author{Yuze Zhou}
\usepackage{amsmath}
\usepackage{amsfonts}
\usepackage{graphicx}
\begin{document}
\section{Dual Formulation of Elastic Net}
\paragraph{}First, to write the optimization problem of elastic net in a matrix form, and denote $D =[0, I]$, the optimization problem becomes:
\begin{center}
$\hat{\beta} = \arg \min\limits_{\beta} \frac{1}{2}||y-X\beta||_{2}^{2} + \lambda_{1}||D\beta||_{1}+\lambda_{2}||D\beta||_{2}^{2}$
\end{center}
\paragraph{}The augmented Lagrangian for the problem is:
\begin{center}
$L = \frac{1}{2}||y-z||_{2}^{2} + \lambda_{1}||\omega||_{1}+\lambda_{2}||\omega||_{2}^{2} + u^{\tau}(z-X\beta) + v^{\tau}(\omega-D\beta)$
\end{center}
\paragraph{}By taking the derivatives with respect to $z$ and $\beta$, we could obtain:
\begin{center}
$0 = \frac{\partial L}{\partial z} = z-y+u$\\
$0 = \frac{\partial L}{\partial \beta} = -X^{\tau}u-D^{\tau}v$
\end{center}
\paragraph{}Since the first column of $X$ is filled with ones and the first column of $D$ is filled with zeros, the first row of $-X^{\tau}u-D^{\tau}v$ gives $\textbf{1}^{\tau} u = 0$. Moreover, since the rest dimensions of $\omega$ is penalized element-wisely in the augmented Lagrangian, we can minimize over $\omega$ by minimizing over each $\omega_{i}, \quad i \geq 2$, that is, we have to minimize $\lambda_{1}|\omega_{i}| + \lambda_{2}\omega_{i}^{2} - v^{\tau}X_{i}\omega_{i}$ for each dimension of $\omega$, where $X_{i}$ denotes the $i$th column of $X$, therefore:
\begin{center}
$\min\limits_{\omega_{i}} \lambda_{1}|\omega_{i}| + \lambda_{2}\omega_{i}^{2} - v^{\tau}X_{i}\omega_{i} $\\
$ $\\
$= \left\{
\begin{aligned}
0 \quad if \quad |v^{\tau}X_{i}| \leq \lambda_{1}\\
-\frac{(\lambda_{1}-|v^{\tau}X_{i}|)^{2}}{4\lambda_{2}} \quad if \quad |v^{\tau}X_{i}| > \lambda_{1}\\
\end{aligned}
\right.
$
\end{center}
\paragraph{}By taking all the above back to the Lagrangian, we obtain the dual problem as:
\begin{center}
$d^{*} = \min\limits_{u} \frac{1}{2}||y-u||_{2}^{2} + \sum\limits_{j: |X_{j}^{\tau}u| > \lambda_{1}}\frac{(\lambda_{1}-|u^{\tau}X_{i}|)^{2}}{4\lambda_{2}}$\\
$subject \quad to \quad \textbf{1}^{\tau}u = 0$
\end{center}
\paragraph{}First denote $f(u) = \sum\limits_{j: |X_{j}^{\tau}u| > \lambda_{1}}\frac{(\lambda_{1}-|u^{\tau}X_{i}|)^{2}}{4\lambda_{2}}$, clearly $f(u)$ is of quadratic form, $f(u) = \frac{1}{2}u^{\tau}Au+a^{\tau}u+b$, where $b$ is a constant and does not matter in the optimization of the dual problem, $A$ and $a$ are:
\begin{center}
$A = \frac{1}{2\lambda_{2}}X_{E}X_{E}^{\tau}$\\
$E := \{i: |X_{i}^{\tau}u| > \lambda \}$
\end{center}
\begin{center}
$a = \frac{\lambda_{1}}{2\lambda_{2}}(\sum\limits_{i:X_{i}^{\tau}<-\lambda_{1}}X_{i}-\sum\limits_{i:X_{i}^{\tau}>\lambda_{1}}X_{i})$
\end{center}
\paragraph{}The dual problem could also be written in a proximal form:
\begin{center}
$\hat{u} = \textbf{prox}_{\tilde{f}}(y)$\\
$\tilde{f} = \textbf{I}(\textbf{1}^{\tau}u=0)f(u) + \textbf{I}(\textbf{1}^{\tau}u \neq 0)\infty$
\end{center}
\paragraph{}After transforming $f(u)$ into a quadratic form, we could write the Lagrangian for the dual problem back again:
\begin{center}
$L = \frac{1}{2}||y-u||_{2}^{2} + \frac{1}{2}u^{\tau}Au+a^{\tau}u+b + \lambda\textbf{1}^{\tau}u$
\end{center}
\paragraph{}By taking the derivative with respect to $u$, we could obtain:
\begin{center}
$\frac{\partial L}{ \partial u} = u - y +Au+a+\lambda\textbf{1} = 0$
\end{center}
\paragraph{}By shifting the terms, $u$ could be written as a formula of $y$ and $\lambda$: $u = (I+A)^{-1}(y-a-\lambda\textbf{1})$, by taking the derivative with respect to $y$ at both sides, we could obtain the Jacobian matrix $J$ of the proximal operator $\textbf{prox}(\tilde{f})$ at $y$ as:
\begin{center}
$J = (I+A)^{-1} -(I+A)^{-1}\textbf{1}\nabla(\hat{\lambda})^{\tau}$
\end{center}
where $\nabla(\hat{\lambda})^{\tau}$ denotes the gradient of $\lambda$ as a function of $y$.
\paragraph{}By taking $u = (I+A)^{-1}(y-a-\lambda\textbf{1})$ back to the Lagrangian, the dual problem will become a second-order equation of $\lambda$:
\begin{center}
$d^{*} = \max\limits_{\lambda} \frac{1}{2}||y-(I+A)^{-1}(y-a-\lambda\textbf{1})||_{2}^{2}+\frac{1}{2}(y-a-\lambda\textbf{1})^{\tau}(I+A)^{-1}A(I+A)^{-1}(y-a-\lambda\textbf{1})+a^{\tau}(I+A)^{-1}(y-a-\lambda\textbf{1})+\lambda*\textbf{1}^{\tau}(I+A)^{-1}(y-a-\lambda\textbf{1})$
\end{center}
\paragraph{}More specifically, the second-order term is:
\begin{center}
$\frac{1}{2}\textbf{1}^{\tau}(I+A)^{-2}\textbf{1}+\frac{1}{2}\textbf{1}^{\tau}(I+A)^{-1}A(I+A)^{-1}\textbf{1}-\textbf{1}^{\tau}(I+A)^{-1}\textbf{1}$
\end{center}
and the first-order term is:
\begin{center}
$2\textbf{1}^{\tau}(I+A)^{-1}(y-a)-\textbf{1}^{\tau}(I+A)^{-2}(y-a)-\textbf{1}^{\tau}(I+A)^{-1}A(I+A)^{-1}(y-a)$
\end{center}
\paragraph{}Thus by solving the second-order equation, we could obtain
\begin{center}
$\hat{\lambda} = \frac{2\textbf{1}^{\tau}(I+A)^{-1}(y-a)-\textbf{1}^{\tau}(I+A)^{-2}(y-a)-\textbf{1}^{\tau}(I+A)^{-1}A(I+A)^{-1}(y-a)}{\textbf{1}^{\tau}(I+A)^{-2}\textbf{1}+\textbf{1}^{\tau}(I+A)^{-1}A(I+A)^{-1}\textbf{1}-2\textbf{1}^{\tau}(I+A)^{-1}\textbf{1}}$
\end{center}
and the gradient
\begin{center}
$\nabla(\hat{\lambda}) = \frac{2(I+A)^{-1}\textbf{1}-(I+A)^{-2}\textbf{1}-(I+A)^{-1}A(I+A)^{-1}\textbf{1}}{\textbf{1}^{\tau}(I+A)^{-2}\textbf{1}+\textbf{1}^{\tau}(I+A)^{-1}A(I+A)^{-1}\textbf{1}-2\textbf{1}^{\tau}(I+A)^{-1}\textbf{1}}$
\end{center}
\paragraph{}By taking the gradient back to $J = (I+A)^{-1} -(I+A)^{-1}\textbf{1}\nabla(\hat{\lambda})^{\tau}$, we could obtain the Jacobian.
\end{document}
